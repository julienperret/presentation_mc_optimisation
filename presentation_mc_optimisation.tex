\documentclass{beamer}
\usepackage[utf8]{inputenc}

\title{Optimisation stochastique, (RJ)MCMC et applications g\'eospatiales}
\author{}
\institute{IGN}
\date{2015}

\begin{document}
\frame{\titlepage}
 
\begin{frame}
\frametitle{Contexte}
\begin{itemize}
\item La recherche de l'IGN a souvent des probl\`emes d'optimisation (comme tout le monde)
\item D\'eveloppement d'outils \emph{sp\'ecifiques} (g\'en\'eralisation carto)
\item D\'eveloppement d'un biblioth\`eque plus \emph{g\'en\'erique}
\end{itemize}
\end{frame}

\begin{frame}
\frametitle{La \emph{libRJMCMC}}
Elle propose~:
\begin{itemize}
\item un \'echantillonneur RJMCMC
\item un recuit simul\'e (ou parallel tempering)
\item des exemples et des applications
\end{itemize}
Plusieurs impl\'ementations~:
\begin{itemize}
\item C++ (+ lien)
\item Java (+ lien)
\item Scala (+ lien)
\end{itemize}
Le tout sous licence libre.
\end{frame}

\section{Applications}
\begin{frame}
\frametitle{Simplu3d}
\emph{Le probl\`eme~:} simulation des droits \`a batir
\emph{Les concepts~:}
\begin{itemize}
\item MPP~:
\item A prioris~:
\item Noyaux~:
\item \'Energie~:
\end{itemize}
\emph{Difficultés/verrous~:}prise en compte de l'impr\'ecision spatiale
\end{frame}

\begin{frame}
\frametitle{Simplu3d}
\emph{R\'esultats:}
\end{frame}

\end{document}
